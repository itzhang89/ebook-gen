% define chinese newline rules

\XeTeXlinebreaklocale "zh"
\XeTeXlinebreakskip = 0pt plus 1pt minus 0.1pt

% define margion size

\usepackage[top=2cm, bottom=1.5cm, left=2cm, right=2cm]{geometry}

% set fonts
\usepackage{xeCJK}                      %使用xeCJK宏包
% 英文字体配置部分
%\setmainfont{Source Serif Pro}%Times New Roman
%\setsansfont{Source Sans Pro}
%\setmonofont{Source Code Pro}
% 中文字体配置部分
%\usepackage{xeCJK}%中文字体
%\setCJKmainfont{Songti SC}%宋体正文字体
%\setCJKsansfont{PingFang SC}%黑体 无衬线字体
%\setCJKmonofont{Kaiti SC}%楷体 等宽字体
%\setCJKfamilyfont{boldsong}{Source Han Serif SC Heavy}
%\setCJKmainfont{FandolSong}%正文字体
%\setCJKsansfont{FandolHei}%无衬线字体
%\setCJKmonofont{楷体}%等宽字体
%\setCJKfamilyfont{Regular}{FandolFang}

% Start a new page only after TOC

\let\oldtoc\tableofcontents
\renewcommand{\tableofcontents}{\oldtoc\newpage}

% change background color for inline code in
% markdown files. The following code does not work well for
% long text as the text will exceed the page boundary
\definecolor{bgcolor}{HTML}{E0E0E0}
\let\oldtexttt\texttt

\renewcommand{\texttt}[1]{
    \colorbox{bgcolor}{\oldtexttt{#1}}
}

% Change the default style of block quote

\usepackage{framed}
\usepackage{quoting}

\definecolor{bgcolor}{HTML}{DADADA}
\colorlet{shadecolor}{bgcolor}
% define a new environment shadedquotation. You can change leftmargin and
% rightmargin as you wish.
\newenvironment{shadedquotation}
{\begin{shaded*}
     \quoting[leftmargin=1em, rightmargin=0pt, vskip=0pt, font=itshape]
     }
     {\endquoting
\end{shaded*}
}

%
\def\quote{\shadedquotation}
\def\endquote{\endshadedquotation}


